%%%%%%%%%%%%%%%%%%%%%%%%%%%%%%%%%%%%%%%%%%%%%%%%%%%%%%%%%%%%%%%%%%%%%%%%%%%%%%%
% Appendix A - Derivation of the Stern-Volmer Relationship
%%%%%%%%%%%%%%%%%%%%%%%%%%%%%%%%%%%%%%%%%%%%%%%%%%%%%%%%%%%%%%%%%%%%%%%%%%%%%%%

\chapter{Derivation of the Stern-Volmer Relationship}

Upon absorption of radiation, a phosphorescent molecule is excited (Figure \ref{fig:jablonski}A) from its singlet ground state ($S_0$) to an excited singlet state ($S_1$) while maintaining the pairing of its electron spins. The excited singlet state can then non-radiatively transition to an excited triplet state ($T_1$) through a process known as intersystem crossing. This results in the unpairing of its ground and excited state electron spins.

The excited triplet state molecule can return back to its singlet ground state through either radiative or non-radiative relaxation. The radiative relaxation from the excited triplet state to the singlet ground state is known as phosphorescence and occurs with a decay rate $k_{r}$. Non-radiative relaxation occurs through intersystem crossing at a decay rate $k_{n-r}$. Collision with ground triplet state molecular oxygen (\ce{^3O2}) can quench the excited triplet state molecule back to its singlet ground state while producing excited singlet state oxygen (\ce{^1O2}). This transfer of electronic excitation energy can be modeled as a first order reaction with a decay rate proportional to the product of the quenching constant ($k_{q}$) and the concentration of dissolved oxygen ([\ce{O2}]).
%
The overall rate equation can be expressed as:
%
\begin{equation}
    \frac{d[P^*]}{dt} = -(k_{r} + k_{n-r} + k_{q}[O_{2}])[P^{*}]
\end{equation}
%
where $[P^{*}]$ denotes the number of excited triplet state molecules. Assuming that $[O_{2}] >> [P^{*}]$, then the rate equation can be integrated to:
%
\begin{equation}
    [P^*] = -[P^{*}]_{0}e^{-(k_{r} + k_{n-r} + k_{q}[O_{2}])t}
\end{equation}
%
The lifetime in the presence of the quencher ($\tau$) can then be defined by inversion of the overall decay rate:
%
\begin{equation}
    \tau = \frac{1}{k_{r} + k_{n-r} + k_{q}[O_{2}]}
\end{equation}
%
In the absence of the quencher ($[O_2] = 0$), the lifetime can be simplified to:
%
\begin{equation}
    \tau_{0} = \frac{1}{k_{r} + k_{n-r}}
\end{equation}
%
and used to normalize the lifetime in the presence of the quencher:
%
\begin{equation}
    \frac{\tau_{0}}{\tau} = \frac{k_{r} + k_{n-r} + k_{q}[O_{2}]}{k_{r} + k_{n-r}}
\end{equation}
%
This can be simplified to a linear expression known as the Stern-Volmer Relationship:
%
\begin{equation}
    \frac{\tau_{0}}{\tau} = 1 + k_{q}\tau_{0}[O_{2}]
\end{equation}
%
Using Henry's Law for dissolved gases, $pO_2 = K_H[O_2]$, the molecular oxygen concentration [\ce{O2}] can be exchanged for the partial pressure of oxygen (\ce{pO2}):
%
\begin{equation}
    \frac{\tau_{0}}{\tau} = 1 + k_{q}\tau_{0}[pO_{2}]
\end{equation}
%
Lifetime can be calibrated against a standard that quantifies either the [\ce{O2}] or \ce{pO2} in a solution that matches the pH, atmospheric pressure, temperature, and salinity of the desired sample environment.
