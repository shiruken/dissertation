%%%%%%%%%%%%%%%%%%%%%%%%%%%%%%%%%%%%%%%%%%%%%%%%%%%%%%%%%%%%%%%%%%%%%%%%%%%%%%%
% Chapter 6 - Conclusions
%%%%%%%%%%%%%%%%%%%%%%%%%%%%%%%%%%%%%%%%%%%%%%%%%%%%%%%%%%%%%%%%%%%%%%%%%%%%%%%

\chapter{Conclusions} \label{Chapter_6}

Optical imaging technologies have been utilized extensively in preclinical research to study the complex hemodynamics of the normal and diseased brain. They offer significant advantages over many other imaging modalities because of superior spatial and temporal resolutions, the availability of a wide array of contrast agents, and relatively inexpensive hardware costs. The distinct wavelengths and spectra of illuminating and detected light readily facilitates the combination of different imaging modalities. This permits simultaneous multi-parameter measurements in subjects without requiring repeated trials for each individual imaging modality and provides a more comprehensive understanding of cerebral hemodynamics.

This dissertation presented the development of a dual-modality optical imaging platform capable of acute and chronic monitoring of cerebral hemodynamics during ischemic stroke and the subsequent recovery. The system combined laser speckle contrast imaging (LSCI) with oxygen-dependent quenching of phosphorescence to measure relative blood flow and oxygen tension within cortical vasculature with high spatial and temporal resolutions. A digital micromirror device (DMD) was used to spatially-pattern phosphorescent excitation light to overcome traditional limitations of the imaging technique. The DMD was also utilized to perform spatially-targeted photothrombosis within pial arterioles for the induction of highly-localized ischemic lesions. The capabilities of the imaging system were demonstrated by monitoring the acute hemodynamic response during targeted photothrombosis and the chronic recovery of the resulting infarct. The LSCI hardware and software was upgraded to perform multi-exposure speckle imaging (MESI) in order to produce more quantitatively accurate estimates of blood flow via the improved physical model. The transition to fully awake imaging using a head-restrained treadmill eliminated the confounding effects of anesthesia upon cerebral hemodynamics. The capabilities of the complete awake imaging system were again demonstrated by monitoring the acute and chronic response to targeted photothrombosis.



%%%%%%%%%%%%%%%%%%%%%%%%%%%%%%%%%%%%%%%%%%%%%%%%%%%%%%%%%%%%%%%%%%%%%%%%%%%%%%%
% Section 6.1 - Future Work
%%%%%%%%%%%%%%%%%%%%%%%%%%%%%%%%%%%%%%%%%%%%%%%%%%%%%%%%%%%%%%%%%%%%%%%%%%%%%%%
\section{Future Work}

While the dual-modality imaging system is functionally complete, there are numerous improvements that could be made to the instrumentation and acquisition control software. MESI is currently performed using a standalone MATLAB script running in parallel with the Speckle Software, which only controls the camera. Integrating the generation of the necessary trigger/modulation waveforms and the calibration procedure directly into the Speckle Software would greatly simplify MESI acquisitions and facilitate distribution to other researchers. This would also permit for streamlining the signal generation and acquisition of the phosphorescence lifetime measurements so that they are properly interleaved between MESI illumination pulses to minimize crosstalk between the two imaging modalities.

The impact of camera bit depth on the accuracy of MESI estimates of $\tau_c$ have yet to be quantified. Comparisons between 8-bit and 16-bit cameras using traditional single-exposure LSCI found minimal differences in relative flow accuracy \cite{Richards:2013bi, Richards:2016hy} despite the increased sensitivity and dynamic range offered by more expensive cameras. Higher frame rates were found to be more important than bit depth because they allowed for greater temporal averaging. However, because complete MESI frames are generated much more slowly than a single-exposure LSCI frame, it is possible the technique could benefit from the increased sensitivity of a 12- or 16-bit camera.

The current spectral separation of the imaging system offers opportunities for integrating other optical imaging techniques that utilize wavelengths between 475-600 nm. Multispectral reflectance imaging at those wavelengths would allow for detecting relative changes in oxy- and deoxyhemoglobin based on differences in absorption. This information can be used in conjunction with LSCI measurements of blood flow to estimate the cerebral metabolic rate of oxygen consumption (\ce{CMRO2}), which is widely used in neurovascular studies \cite{Jones:2008gb}. The dual-modality imaging system has already been demonstrated while simultaneously performing spatially resolved electrical recordings of neural activity during and following targeted photothrombosis \cite{Luan:2018fq}. Another potential application of the DMD could be spatially-patterning LSCI illumination light to reconstruct 3D tomographic measurements of flow from the imaged speckle dynamics \cite{Huang:2017gm}.

This dissertation focused primarily on the development of the hardware for the imaging system and not the underlying neuroscience it is capable of probing. There are numerous phenomenon that could be studied using the simultaneous measurements of blood flow and \ce{pO2} or the targeted photothrombotic model of ischemic stroke. Establishing robust baseline measurements of both parameters across various vessels and multiple animals would help facilitate acute and chronic hemodynamic imaging. Because anesthesia has been shown to suppress and delay neurovascular coupling \cite{Pisauro:2013cx}, there is an opportunity to reexamine the metabolic and hemodynamic changes associated with functional activation \cite{Dunn:2005gw} in awake subjects. Anesthesia has also been shown to suppress the occurrence of ischemia-induced spreading depolarizations \cite{Kudo:2016ho}, which have been implicated in expanding the size of the ischemic core \cite{Shin:2006dc}. Awake imaging with MESI would allow for more accurate measurements of ischemic flow deficits and permit a more robust multi-animal study of this phenomenon. Combining the quantitative imaging of hemodynamics with measurements of functional impairment would allow for testing the efficacy of acute pharmaceutical interventions and rehabilitation.



%%%%%%%%%%%%%%%%%%%%%%%%%%%%%%%%%%%%%%%%%%%%%%%%%%%%%%%%%%%%%%%%%%%%%%%%%%%%%%%
% END Chapter 6
%%%%%%%%%%%%%%%%%%%%%%%%%%%%%%%%%%%%%%%%%%%%%%%%%%%%%%%%%%%%%%%%%%%%%%%%%%%%%%%
