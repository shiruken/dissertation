%%%%%%%%%%%%%%%%%%%%%%%%%%%%%%%%%%%%%%%%%%%%%%%%%%%%%%%%%%%%%%%%%%%%%%%%%%%%%%%
% Chapter 3 - Spatially-Targeted Photothrombotic Stroke
%%%%%%%%%%%%%%%%%%%%%%%%%%%%%%%%%%%%%%%%%%%%%%%%%%%%%%%%%%%%%%%%%%%%%%%%%%%%%%%

\chapter{Spatially-Targeted Photothrombotic Stroke} \label{ch:photothrombosis}

Animal models of ischemic stroke are extensively used to study the mechanisms of neuronal death and recovery and to perform preliminary testing on neuroprotective interventions. While there are numerous techniques for inducing focal ischemia, the majority rely upon occlusion of the middle cerebral artery (MCA) and its branches. The MCA is the largest cerebral artery in the brain and the most common vessel involved with human ischemic events \cite{Sicard:2009ku}. The models that can most reliably reproduce the lesions and pathophysiology of human stroke (e.g. ischemic core and penumbra) offer the best experimental platforms for preclinical research.

Intraluminal MCA occlusion (MCAo) is the most widely used technique and is performed by introducing a monofilament suture into the internal carotid artery to block blood flow to the MCA \cite{Kozuimi:1986bd}. This model is capable of inducing both permanent and transient focal ischemia similar to that of human stroke and does not require craniotomy. The procedure results in large-scale infarct volumes (21-45\% of ipsilateral hemisphere) that most closely resemble malignant infarction in humans \cite{Carmichael:2005gk}. However, the majority of human strokes are much smaller in size (4.5-14\%) \cite{Carmichael:2005gk, Brott:1989bl}, making traditional MCAo a poor model for studying recovery at a similar scale. Distal MCAo produces smaller infarcts limited to the cerebral hemisphere but requires performing a craniotomy to physically access the target vessel \cite{Doyle:2014bz}. Embolic MCAo relies upon the introduction of microspheres or the induction of thrombotic clots to occlude downstream vasculature. Particle size dictates the extent and localization of the infarction, which are more variable than traditional or distal MCAo \cite{Carmichael:2005gk}. Vasoconstrictors such as Endothelin-1 (ET-1) can be injected intracerebrally in the proximity of the MCA to induce transient ischemia with a dose-dependent recovery of blood flow \cite{Sicard:2009ku}. Surgical electrocauterization or direct clipping of the MCA can also be performed to induce permanent or reversible ischemia but require craniotomy.

The photothrombosis model uses intravascular photooxidation to generate well-defined cortical lesions \cite{Watson:1985bp}. Photosensitive dyes such as rose bengal are injected intravenously and irradiated with light to produce singlet oxygen, which causes localized endothelial damage initiating platelet aggregation and thrombus formation \cite{Dietrich:1987wh}. Rose bengal has been extensively utilized as a photothrombotic agent \cite{Grome:1988bx, Parthasarathy:2010vo} and has well-characterized pharmacokinetics with fast clearance from the body \cite{Klaassen:1976kg}. A significant advantage of the photothrombotic model is the ability to stereotactically control the position and size of the infarct to target specific functional regions. However, the technique results in rapid vasogenic edema, which is thought to restrict the development of the ischemic penumbra and local reperfusion \cite{Carmichael:2005gk}.

The DMD in the imaging system offers a new method for targeting photothrombosis that allows for increased control over the stroke induction process compared to conventional techniques that only illuminate a single focal volume. Entire vessels, arbitrarily-shaped regions, or even multiple locations can be simultaneously occluded by using the DMD to pattern the irradiating light. By specifically targeting vessels, collateral photooxidative damage to the surrounding tissue can be minimized. This chapter details modifications made to the imaging system to perform DMD-targeted photothrombosis and an \textit{in vivo} demonstration of the technique in mice. The targeted photothrombosis technique was described in a methods paper published in \textit{Neurophotonics} \cite{Sullender:2018ff}.



%%%%%%%%%%%%%%%%%%%%%%%%%%%%%%%%%%%%%%%%%%%%%%%%%%%%%%%%%%%%%%%%%%%%%%%%%%%%%%%
% Section 3.1 - Instrumentation Modifications
%%%%%%%%%%%%%%%%%%%%%%%%%%%%%%%%%%%%%%%%%%%%%%%%%%%%%%%%%%%%%%%%%%%%%%%%%%%%%%%
\section{Instrumentation Modifications}

The system was modified (Figure \ref{fig:systemschematic_2}) to perform photothrombotic stroke with the addition of a 532 nm laser (200 mW, AixiZ LLC). The packaged diode laser has a 2 mm collimated output that operates at a fixed current with convection cooling. A neutral density filter (OD 1.0, NE10A-A, Thorlabs, Inc.) was used to attenuate the laser intensity by an order of magnitude down to 20 mW. A longpass dichroic beamsplitter (490 nm cutoff, DMLP490, Thorlabs, Inc.) was used to coalign the 532 nm laser with the other two lasers for coupling into the fiber optic patch cord. The green light can then be patterned by the DMD for targeted photothrombosis. \ce{pO2} measurements can be acquired simultaneously during photothrombosis induction because of the system's spectral separation, but are limited to only the stroke target region.

% Figure - System Schematic (Ver. 2)
\begin{figure}
    \includegraphics{figures/chapter_3/systemschematic_2.pdf}
    \caption{
        \label{fig:systemschematic_2}
        The optical system was modified with the addition of a 532 nm laser coupled into the fiber optic patch cord for DMD-targeted photothrombosis.
    }
\end{figure}



%%%%%%%%%%%%%%%%%%%%%%%%%%%%%%%%%%%%%%%%%%%%%%%%%%%%%%%%%%%%%%%%%%%%%%%%%%%%%%%
% Section 3.2 - Targeted Photothrombosis Induction
%%%%%%%%%%%%%%%%%%%%%%%%%%%%%%%%%%%%%%%%%%%%%%%%%%%%%%%%%%%%%%%%%%%%%%%%%%%%%%%
\section{Targeted Photothrombosis Induction} \label{sec:photothrombosis_induction}

Targeted photothrombosis was demonstrated \textit{in vivo} using anesthetized (1.5\% isoflurane in medical air) mice with permanent cranial window implants. Rose bengal was administered intravenously via retro-orbital injection (50 $\mu$L, 15 mg/mL) and the subject was immediately exposed to DMD-patterned green light for 5-10 minutes. Descending arterioles were the primary targets because they serve as bottlenecks in the cortical oxygen supply \cite{Nishimura:2007hk}. Target vessels were identified based on vascular orientation and \textit{a posteriori} knowledge. Because Oxyphor PtG4 has minimal absorbance of green light, \ce{pO2} measurements can be simultaneously acquired while performing photothrombosis. However, the measurements were limited to only the region being targeted for occlusion. LSCI was used to monitor clot formation within the targeted area and to control the progression of the occlusion. The open source image registration software \texttt{elastix} \cite{Klein:2010gr} was used during post-processing to correct the speckle contrast images for any motion relative to the beginning of the acquisition. This would infrequently occur if the animal was not properly secured in the stereotaxic frame. The automatic intensity-based transform allowing for rotation and translation was applied to each speckle contrast frame. The computed speckle contrast inverse correlation times (ICT = $1/\tau_c$) were then baselined against pre-stroke values to provide an estimate of the relative change in blood flow (rICT = $\tau_{c,initial}/\tau_c$).

Figure \ref{fig:photothrombosisacute} depicts the targeted photothrombotic occlusion of a descending arteriole that is likely a distal branch of the MCA. The red overlay in Figure \ref{fig:photothrombosisacute}A highlights the 0.09 mm$^2$ region irradiated with spatially-patterned green light for 420 seconds. \ce{pO2} measurements were continuously acquired from the same region at 1 Hz with 2500 decays averaged per record. The series of speckle contrast images depict the progression of the photothrombotic occlusion as the targeted vessel rapidly underwent stenosis and flow was significantly reduced. After only two minutes of exposure, the targeted vessel was indistinguishable from the surrounding parenchyma.

% Figure - Acute Photothrombosis
\begin{figure}
    \includegraphics{figures/chapter_3/photothrombosisacute.pdf}
    \caption[\textbf{(A)} Speckle contrast images depicting the occlusion of a descending arteriole using DMD-targeted photothrombosis. The red overlay indicates the 0.09 mm$^2$ region simultaneously illuminated for occlusion and \ce{pO2} measurements. \textbf{(B)} Two arterioles (A1, A2), one vein (V1), and two parenchyma regions (P1, P2) were targeted for \ce{pO2} measurements after stroke induction. \textbf{(C)} Relative blood flow and \ce{pO2} within the targeted regions during and after photothrombosis. The green-shaded section indicates irradiation of the targeted arteriole. The arrow indicates the propagation of an ischemia-induced depolarization event (Scale bars = 1 mm).]{
        \label{fig:photothrombosisacute}
        \textbf{(A)} Speckle contrast images depicting the occlusion of a descending arteriole using DMD-targeted photothrombosis. The red overlay indicates the 0.09 mm$^2$ region simultaneously illuminated for occlusion and \ce{pO2} measurements. \textbf{(B)} Two arterioles (A1, A2), one vein (V1), and two parenchyma regions (P1, P2) were targeted for \ce{pO2} measurements after stroke induction. \textbf{(C)} Relative blood flow and \ce{pO2} within the targeted regions during and after photothrombosis. The green-shaded section indicates irradiation of the targeted arteriole. The arrow indicates the propagation of an ischemia-induced depolarization event (Scale bars = 1 mm). Adapted from \cite{Sullender:2018ff}.
    }
\end{figure}

% Figure - PID Propagation
\begin{figure}
    \includegraphics{figures/chapter_3/pidsequence.pdf}
    \caption{
        \label{fig:pidsequence}
        Relative flow during an ischemia-induced depolarization event that results in the expansion of the flow deficit. The depolarization has minimal effect upon the large draining vein. (Scale bar = 1 mm).
    }
\end{figure}

Figure \ref{fig:photothrombosisacute}B depicts the five regions (two arterioles, one vein, two parenchyma) targeted for dynamic relative blood flow and \ce{pO2} measurements. The first arteriole region (A1) is the same vessel targeted for photothrombotic occlusion. The resulting timecourses of relative blood flow and \ce{pO2} within each region can be seen in Figure \ref{fig:photothrombosisacute}C. By $t$ = 120 seconds, flow within the targeted arteriole had decreased to \textless50\% of baseline and \ce{pO2} had dropped from 80 mmHg to only 20 mmHg. Over the following several minutes, flow also decreased in the nearby parenchyma and venous regions (P1 and V1) and increased slightly in the distal second arteriole region (A2). The propagation of an ischemia-induced depolarization event \cite{Shin:2006dc, Dreier:2011gz} can be seen beginning at $t$ = 300 seconds, with sharp reductions in both relative blood flow and \ce{pO2}. As the depolarization subsided, flow within the targeted arteriole further decreased to \textless35\% of baseline flow while the \ce{pO2} returned to pre-depolarization levels around 20 mmHg. Flow in all other regions remained depressed immediately following the depolarization. Figure \ref{fig:pidsequence} overlays relative ICT on speckle contrast imagery to depict the spatial extent of the depolarization and the resulting increase in deficit area. This global reduction in flow following a spreading depolarization is consistent with previous studies using other stroke models \cite{Shin:2006dc, Nakamura:2010wp}.

Photothrombosis irradiation was stopped at $t$ = 420 seconds and \ce{pO2} measurements from all five ROIs were initiated. Patterns were displayed at 2 Hz with 1250 decays averaged per record. Flow and \ce{pO2} decreased over the remaining 20 minutes of the imaging session across all regions except for A1. At $t$ = 860 seconds, the targeted vessel partially reperfused, causing an abrupt increase in both relative blood flow (+6 percentage points) and \ce{pO2} (+15 mmHg). By the end of the imaging session, flow had increased within A1 to 55\% of baseline and \ce{pO2} to 42 mmHg, likely indicating further reperfusion of the vessel.



%%%%%%%%%%%%%%%%%%%%%%%%%%%%%%%%%%%%%%%%%%%%%%%%%%%%%%%%%%%%%%%%%%%%%%%%%%%%%%%
% Section 3.3 - Chronic Post-Stroke Hemodynamics
%%%%%%%%%%%%%%%%%%%%%%%%%%%%%%%%%%%%%%%%%%%%%%%%%%%%%%%%%%%%%%%%%%%%%%%%%%%%%%%
\section{Chronic Post-Stroke Hemodynamics} \label{sec:chronic_hemodynamics}

The chronic progression of the ischemic lesion was monitored for eight days following photothrombosis. The subject was anesthetized (1.5\% isoflurane in medical air) and positioned on the stereotaxic frame such that the speckle contrast imagery was aligned with previous acquisitions. Imaging sessions were kept as short as possible (\textless30 minutes) to avoid over-exposure to isoflurane. The perfusion of the occluded arteriole and broader effects on cortical flow were tracked using LSCI as shown in Figure \ref{fig:photothrombosischronic}A. Tiled \ce{pO2} maps acquired using both 445 and 637 nm excitation (Figure \ref{fig:photothrombosischronic}B-C) reveal the spatial extent of the oxygen deficit. The broad value range of the colormap masks the smaller \ce{pO2} differences between arterioles and venules. A large gradient can be seen between the occluded vessel and surrounding tissue on Days +1 and +2 despite partial reperfusion of the targeted vessel. This gradient resembles the ischemic penumbra that the photothrombotic technique rarely produces \cite{Carmichael:2005gk}. By Day +5, the targeted arteriole had fully reperfused and the hypoxic region recovered to near baseline.

% Figure - Chronic Photothrombosis
\begin{figure}
    \includegraphics{figures/chapter_3/photothrombosischronic.pdf}
    \caption[Progression of the ischemic lesion over eight days as imaged with \textbf{(A)} LSCI, \textbf{(B)} 445 nm tiled \ce{pO2}, and \textbf{(C)} 637 nm tiled \ce{pO2} measurements. Day -0 measurements were taken immediately prior to photothrombosis induction and Day +0 measurements were taken immediately after. \textbf{(D)} Two arterioles (A1, A2), one vein (V1), and two parenchyma regions (P1, P2) were targeted for chronic \textbf{(E)} relative blood flow and \textbf{(F)} 445 nm \ce{pO2} measurements (mean $\pm$ s.d.). The relative blood flow was baselined against Day -0 measurements. (Scale bars = 1 mm).]{
        \label{fig:photothrombosischronic}
        Progression of the ischemic lesion over eight days as imaged with \textbf{(A)} LSCI, \textbf{(B)} 445 nm tiled \ce{pO2}, and \textbf{(C)} 637 nm tiled \ce{pO2} measurements. Day -0 measurements were taken immediately prior to photothrombosis induction and Day +0 measurements were taken immediately after. \textbf{(D)} Two arterioles (A1, A2), one vein (V1), and two parenchyma regions (P1, P2) were targeted for chronic \textbf{(E)} relative blood flow and \textbf{(F)} 445 nm \ce{pO2} measurements (mean $\pm$ s.d.). The relative blood flow was baselined against Day -0 measurements. (Scale bars = 1 mm). Adapted from \cite{Sullender:2018ff}.
    }
\end{figure}

The same five regions (two arterioles, one vein, and two parenchyma) used during the acute photothrombosis measurements were also targeted for chronic relative blood flow and \ce{pO2} measurements (Figure \ref{fig:photothrombosischronic}D-F). The relative blood flow was calculated using the spatial average of the pre-stroke (Day -0) ICT measurements as the baseline. The first post-stroke measurements (Day +0) were taken immediately after the induction of photothrombosis and revealed global deficits in both blood flow and \ce{pO2}. This systemic change is likely the result of the spreading depolarization, which have previously been shown to cause global reductions in CBF using other stroke models \cite{Shin:2006dc, Nakamura:2010wp}. The blood flow and \ce{pO2} within the ROIs mirror the recovery seen in the spatially-resolved results with the vessel fully reperfusing by Day +5.

The speed of this recovery is faster than previous traditional photothrombotic inductions performed by the lab, which took 3-4 weeks to return to baseline flow levels \cite{Schrandt:2015gu}. This discrepancy can be explained by the smaller area targeted for photothrombosis (0.09 mm$^2$ vs. 0.28 mm$^2$) and the lower irradiance (\textless15 mW/mm$^2$ vs. 72 mW/mm$^2$) within the area of illumination, which resulted in a less severe ischemic lesion. Confinement of the excitation light to only within the targeted arteriole minimized collateral damage to the surrounding vasculature and parenchyma and allowed the infarct to manifest downstream of the occlusion. The lack of reperfusion is also a common critique of the traditional photothrombotic model of stroke \cite{Carmichael:2005gk}.

%%%%%%%%%%%%%%%%%%%%%%%%%%%%%%%%%%%%%%%%%%%%%%%%%%%%%%%%%%%%%%%%%%%%%%%%%%%%%%%
\subsection{Persistence of Oxyphor PtG4 \textit{In Vivo}}

Oxyphor PtG4 is a large molecule with a molecular weight around 35 kDa. The majority of its size arises from the PEGylated hydrophobic dendrimer surrounding the PtTBP core used to increase biocompatibility and lifetime stability. Pilot experiments with Oxyphor PdG4, an analog with a palladium core, found that it was retained in the bloodstream for hours and readily accumulated within tumors via the enhanced permeability and retention effect \cite{Esipova:2011hi}. This allowed animals to be imaged up to a day after being injected with 10 $\mu$M Oxyphor PdG4.

Oxyphor PtG4 persists in the bloodstream significantly longer than previously reported in literature. Strong phosphorescent decay curves have be obtained up to two weeks after an initial injection of the dye with a target blood plasma concentration of 5 $\mu$M. The mouse utilized for the targeted photothrombosis and eight days of chronic hemodynamic imaging described above was only administered a single dose of Oxyphor PtG4 on the first day of imaging. While this allows for efficient usage of a limited supply of dye, it remains unclear how and where the probe persists in the bloodstream for such an extended period of time. Phosphorescent intensity does decrease day-to-day, which means that the probe does not remain indefinitely. However, if the protective dendrimer structure is being degraded, then the reliability of the lifetime measurements is a major concern. Unfortunately, it would be difficult to validate the correct \ce{pO2} using a different technique \textit{in vivo} or to replicate the appropriate testing conditions \textit{in vitro}. In order to mitigate the impact of these uncertainties, supplemental injections of Oxyphor PtG4 for a blood plasma concentration of 5 $\mu$M were administered weekly in subjects being used for chronic imaging.



%%%%%%%%%%%%%%%%%%%%%%%%%%%%%%%%%%%%%%%%%%%%%%%%%%%%%%%%%%%%%%%%%%%%%%%%%%%%%%%
% Section 3.4 - Functional Effects of Targeted Photothrombosis
%%%%%%%%%%%%%%%%%%%%%%%%%%%%%%%%%%%%%%%%%%%%%%%%%%%%%%%%%%%%%%%%%%%%%%%%%%%%%%%
\section{Functional Effects of Targeted Photothrombosis}

A comprehensive comparison between traditional and artery-targeted photothrombosis using the DMD was performed by Taylor A. Clark\footnote{The results of this section are adapted from a manuscript currently under review as `T. Clark, C. Sullender, S. Kazmi, B. Speetles, M. Williamson, D. Palmberg, A. Dunn, and T. Jones. Artery targeted photothrombosis enlarges the vascular penumbra, instigates peri-infarct neovascularization and models upper extremity impairments.' C.S. developed the instrumentation and performed the targeted photothrombosis inductions. T.C. performed the experiments, analyzed the results, and wrote the manuscript.}. This section presents results on the functional impairments caused by targeted photothrombosis in the motor cortex. A total of 23 young-adult (4-6 months) C57/Bl6/YFP-H mice were used to examine the impact on skilled forelimb function, with subjects undergoing targeted photothrombosis ($n$ = 13) or sham procedures ($n$ = 10). Mice with cranial windows were anesthetized with \ce{O2}-vaporized isoflurane (4\% induction, 1.5-2\% maintenance) via nose-cone inhalation and placed in a head-fixed stereotaxic frame. Vitals including oxygen saturation, heart rate, and breath rate were monitored via pulse oximetry and temperature was regulated with a feedback heating pad. Rose bengal was administered via retro-orbital injection (50 $\mu$L, 15 mg/mL) and photothrombosis was initiated after a 30 second delay. Sham subjects received injections of sterile saline. DMD-targeted photothrombosis was performed by irradiating pial arteries supplying the forelimb region of the motor cortex as defined from intracortical mappings \cite{Tennant:2011cx} for 5 minutes. In order to account for variations in the size and caliber of the vessels, a range of 1-3 branches were illuminated with an average total targeted area of 0.15 $\pm$ 0.021 mm$^2$. The primary vessels targeted in all subjects were distal branches of the MCA. Photothrombosis progression was monitored in real-time using LSCI.

%%%%%%%%%%%%%%%%%%%%%%%%%%%%%%%%%%%%%%%%%%%%%%%%%%%%%%%%%%%%%%%%%%%%%%%%%%%%%%%
\subsection{Single Seed Retrieval Task}

The resulting impact on forelimb function was quantified by testing performance during a skilled reaching task. The task was a variation of the single seed retrieval task \cite{Chen:2014hy} where mice are trained to reach for and obtain a millet seed placed on a platform outside of transparent training chamber. A pair of 4 mm wide vertical openings on the left and right sides of the chamber permitted the mice to reach through with only the corresponding paw (Figure \ref{fig:reachingtask}A). The external platform contained three wells at varying distances from each opening for the placement of seeds (Figure \ref{fig:reachingtask}B). Two of the wells were centered on the opening and positioned at distances 3 mm (Position 1) and 7 mm (Position 2) away from the chamber. The third well (Position 3) was placed 2 mm lateral of the distal edge of the opening and 5 mm away from the chamber.

The preferred limb for reaching was determined during shaping by allowing the mice to reach for millet seeds with either limb when seeds were placed immediately outside both openings. The preferred-for-reaching limb was defined as the first one used to make five consecutive reach attempts (Figure \ref{fig:reachingtask}C). For the remaining 2-3 days of the shaping phase, mice were encouraged to reach for a single seed placed in Position 1 with their preferred-for-reaching limb. The shaping phase concluded once mice were able to successfully retrieve the seed 10 times (Figure \ref{fig:reachingtask}D). The mice then underwent 10 consecutive days of training sessions, each comprised of 30 trials. Seeds were placed in one of the three positions per trial, with each position recurring ten times per training session in a randomized order. Mice were permitted four reaching attempts per trial with a successful reaching attempt defined as grasping the seed and bringing it inside the chamber to its mouth. Unsuccessful reaching attempts included missing, displacing, or dropping the seed prior to eating.

% Figure - Reaching Task
\begin{figure}
    \includegraphics{figures/chapter_3/reachingtask.pdf}
    \caption[\textbf{(A)} The transparent training chamber and pair of 4 mm vertical openings used for performing the reaching task. \textbf{(B)} The three positions used for seed placement outside of each opening. \textbf{(C)} A mouse reaching for a seed. \textbf{(D)} Training and measurement timeline for the behavioral experiments.]{
        \label{fig:reachingtask}
        \textbf{(A)} The transparent training chamber and pair of 4 mm vertical openings used for performing the reaching task. Adapted from \cite{Chen:2014hy}. \textbf{(B)} The three positions used for seed placement outside of each opening. \textbf{(C)} A mouse reaching for a seed. \textbf{(D)} Training and measurement timeline for the behavioral experiments.
    }
\end{figure}

The mean asymptotic performance (average of the last two training days) prior to photothrombosis inductions for each of the three positions were 0.44 $\pm$ 0.02 (Position 1), 0.3 $\pm$ 0.02 (Position 2), and 0.2 $\pm$ 0.01 (Position 3). Because performance on Position 3 was much lower compared to the other two positions, it was excluded from analysis. Results were reported as the percent of successful reaches per reaching attempt for Positions 1 and 2. Performance was tested on Days 3, 5, 10, and 20 following photothrombosis induction or sham procedure. The SPSS Statistics (IBM Corp.) software package was used to examine reaching performance between the stroke and sham groups over time using two-way repeated measures ANOVA.

%%%%%%%%%%%%%%%%%%%%%%%%%%%%%%%%%%%%%%%%%%%%%%%%%%%%%%%%%%%%%%%%%%%%%%%%%%%%%%%
\subsection{Tissue Processing and Analysis of Lesion Volume}

Animals were euthanized 30 days after photothrombosis with sodium pentobarbital and transcardially perfused with 0.1 M phosphate buffered saline and 4\% paraformaldehyde. The brains were extracted and stored in 4\% paraformaldehyde for no more than 48 hours before using a vibrating blade microtome (VT1000S, Leica Biosystems) to slice 40 $\mu$m coronal sections spaced 240 $\mu$m apart between approximately 1.34 mm anterior to 0.58 mm posterior to bregma. The sections were Nissl stained in order to identify viable cortical tissue and imaged with 17x magnification. The cortical volume was estimated with Cavalieri's Principle \cite{Rosen:1990bf} using Neurolucida (MBF Bioscience) by computing the product of the summed section areas and the distance between sections. Lesion volume was then calculated as the difference between cortical volumes of the contralateral and ipsilateral hemispheres \cite{Tennant:2011cx}.

%%%%%%%%%%%%%%%%%%%%%%%%%%%%%%%%%%%%%%%%%%%%%%%%%%%%%%%%%%%%%%%%%%%%%%%%%%%%%%%
\subsection{Results}

Artery-targeted photothrombosis in the motor cortex significantly impaired performance on the skilled reaching task compared to the sham control on Days 3, 5, and 10 (Figure \ref{fig:functionalimpairment}A). Two-way repeated measures ANOVA of reaching performance over time revealed a significant effect in the stroke vs. sham grouping ($F_{[1,23]}$ = 145.50, $p$ \textless{} 0.0001) but no significant interaction in grouping over time ($F_{[3,69]}$ = 2.77, $p$ = 0.06). By Day 20, performance on the reaching test in the stroke group was similar to that of the sham group. This transience is likely a byproduct of the relatively small infarcts (Figure \ref{fig:functionalimpairment}B-C). Nevertheless, these results suggest that artery-targeted photothrombosis is suitable for creating reproducible focal lesions in the motor cortex that can be used to model upper extremity impairments.

% Figure - Functional Impairment
\begin{figure}
    \includegraphics{figures/chapter_3/functionalimpairment.pdf}
    \caption{
        \label{fig:functionalimpairment}
        \textbf{(A)} Baseline and post-stroke reaching performance measured by the ratio of successfully retrieved seeds per reaching attempt (mean $\pm$ s.e.). Artery-targeted photothrombosis impaired performance on post-stroke Days 3, 5, and 10 compared to the sham procedure (\textbf{**} $p$ \textless{} 0.001, \textbf{*} $p$ \textless{} 0.02). \textbf{(B)} Averaged lesion reconstructions overlaid on coronal templates. The numbers indicate the anterior-to-posterior coordinates relative to bregma. \textbf{(C)} Volume difference between the contralateral and ipsilateral hemispheres as an estimate of infarct size (mean $\pm$ s.e.).
    }
\end{figure}



%%%%%%%%%%%%%%%%%%%%%%%%%%%%%%%%%%%%%%%%%%%%%%%%%%%%%%%%%%%%%%%%%%%%%%%%%%%%%%%
% Section 3.5 - Discussion
%%%%%%%%%%%%%%%%%%%%%%%%%%%%%%%%%%%%%%%%%%%%%%%%%%%%%%%%%%%%%%%%%%%%%%%%%%%%%%%
\section{Discussion}

The addition of a green laser to the DMD illumination pathway facilitates the induction of arbitrarily-shaped photothrombotic lesions. The system allows for greater control over the spatial characteristics of the induced stroke (e.g. size and location) and permits targeting multiple vessels simultaneously. The creation of an extended occlusion within a single arteriole using targeted photothrombosis was demonstrated for the first time. This is a significant change from existing photothrombotic techniques that generally rely upon broad illumination to occlude a large volume of vasculature \cite{Watson:1985bp, Schrandt:2015gu} or highly-focused light to occlude a single microvessel \cite{Schaffer:2006fb}. The previous iteration of this system could only induce occlusions within a large region relative to the FOV and \ce{pO2} measurements could not be simultaneously acquired \cite{Ponticorvo:2010uv}.

While tissue \ce{pO2} during ischemic depolarizations have been previously examined \cite{vonBornstadt:2015dj}, the results reported in \textit{Neurophotonics} \cite{Sullender:2018ff} represent the first quantification of the acute vascular \ce{pO2} response to a depolarization event and the first chronic hemodynamic tracking of an ischemic infarct. The depolarization resulted in a global flow reduction across all regions, which is consistent with previous reports using other stroke models \cite{Shin:2006dc,Nakamura:2010wp}. Within the targeted arteriole, the blood flow reduction is of greater magnitude (-58\%) than the corresponding decrease in \ce{pO2} (-44\%), which eventually recovers to slightly above pre-depolarization levels. Unfortunately, it is difficult to predict how the \ce{pO2} responded to the depolarization across the other regions, but it likely mirrored the LSCI results. The chronic measurements revealed a rapid recovery over the course of five days with the spatial extent of the ischemic lesion clearly visible on the first two days post-stroke. Functional tests revealed that the artery-targeted photothrombotic model of ischemic stroke could be used to produce detectable and reproducible impairments in upper extremity motor skills. The deficits recovered over the course of several weeks despite the relatively minor lesions.



%%%%%%%%%%%%%%%%%%%%%%%%%%%%%%%%%%%%%%%%%%%%%%%%%%%%%%%%%%%%%%%%%%%%%%%%%%%%%%%
% END Chapter 3
%%%%%%%%%%%%%%%%%%%%%%%%%%%%%%%%%%%%%%%%%%%%%%%%%%%%%%%%%%%%%%%%%%%%%%%%%%%%%%%
