%%%%%%%%%%%%%%%%%%%%%%%%%%%%%%%%%%%%%%%%%%%%%%%%%%%%%%%%%%%%%%%%%%%%%%
% Create report using UT dissertation style
%%%%%%%%%%%%%%%%%%%%%%%%%%%%%%%%%%%%%%%%%%%%%%%%%%%%%%%%%%%%%%%%%%%%%%
\documentclass[12pt]{report}
\usepackage{utdiss2}


%%%%%%%%%%%%%%%%%%%%%%%%%%%%%%%%%%%%%%%%%%%%%%%%%%%%%%%%%%%%%%%%%%%%%%
% Load the necessary packages
%%%%%%%%%%%%%%%%%%%%%%%%%%%%%%%%%%%%%%%%%%%%%%%%%%%%%%%%%%%%%%%%%%%%%%
\usepackage{amsmath,amsthm,amsfonts,amscd}
\usepackage{eucal}
\usepackage{verbatim}
\usepackage{makeidx}
\usepackage{epsfig}
\usepackage{cite}
\usepackage{url}
\usepackage[version=4]{mhchem}

\usepackage[english]{babel}
\usepackage{blindtext}

%%%%%%%%%%%%%%%%%%%%%%%%%%%%%%%%%%%%%%%%%%%%%%%%%%%%%%%%%%%%%%%%%%%%%%
% Required personal information
%%%%%%%%%%%%%%%%%%%%%%%%%%%%%%%%%%%%%%%%%%%%%%%%%%%%%%%%%%%%%%%%%%%%%%
\author{Colin Tan Sullender}
\address{}
\title{Writing a Doctoral Dissertation with \LaTeX{} at
		the University of Texas at Austin}


%%%%%%%%%%%%%%%%%%%%%%%%%%%%%%%%%%%%%%%%%%%%%%%%%%%%%%%%%%%%%%%%%%%%%%
% Required degree and dissertation information
%%%%%%%%%%%%%%%%%%%%%%%%%%%%%%%%%%%%%%%%%%%%%%%%%%%%%%%%%%%%%%%%%%%%%%
\previousdegrees{Ph.D.}
\graduationmonth{August}
\graduationyear{2018}
\typist{the author}


%%%%%%%%%%%%%%%%%%%%%%%%%%%%%%%%%%%%%%%%%%%%%%%%%%%%%%%%%%%%%%%%%%%%%%
% Define names of supervisor(s) and committee members
%%%%%%%%%%%%%%%%%%%%%%%%%%%%%%%%%%%%%%%%%%%%%%%%%%%%%%%%%%%%%%%%%%%%%%
\supervisor
	{Andrew K. Dunn}

\committeemembers
    [Theresa A. Jones]
    [Chong Xie]
    [James W. Tunnell]
    {Ming-Chieh Ding}


%%%%%%%%%%%%%%%%%%%%%%%%%%%%%%%%%%%%%%%%%%%%%%%%%%%%%%%%%%%%%%%%%%%%%%
% Some optional commands to change the document's defaults.	     %
%%%%%%%%%%%%%%%%%%%%%%%%%%%%%%%%%%%%%%%%%%%%%%%%%%%%%%%%%%%%%%%%%%%%%%
%
%\singlespacing
%\oneandonehalfspacing

%\singlespacequote
\oneandonehalfspacequote

\topmargin 0.125in	% Adjust this value if the PostScript file output
			% of your dissertation has incorrect top and
			% bottom margins. Print a copy of at least one
			% full page of your dissertation (not the first
			% page of a chapter) and measure the top and
			% bottom margins with a ruler. You must have
			% a top margin of 1.5" and a bottom margin of
			% at least 1.25". The page numbers must be at
			% least 1.00" from the bottom of the page.
			% If the margins are not correct, adjust this
			% value accordingly and re-compile and print again.
			%
			% The default value is 0.125"

		% If you want to adjust other margins, they are in the
		% utdiss2-nn.sty file near the top. If you are using
		% the shell script Makediss on a Unix/Linux system, make
		% your changes in the utdiss2-nn.sty file instead of
		% utdiss2.sty because Makediss will overwrite any changes
		% made to utdiss2.sty.

%%%%%%%%%%%%%%%%%%%%%%%%%%%%%%%%%%%%%%%%%%%%%%%%%%%%%%%%%%%%%%%%%%%%%%
% Some optional commands to be tested.				     %
%%%%%%%%%%%%%%%%%%%%%%%%%%%%%%%%%%%%%%%%%%%%%%%%%%%%%%%%%%%%%%%%%%%%%%

% If there are 10 or more sections, 10 or more subsections for a section,
% etc., you need to make an adjustment to the Table of Contents with the
% command \longtocentry.
%
%\longtocentry



%%%%%%%%%%%%%%%%%%%%%%%%%%%%%%%%%%%%%%%%%%%%%%%%%%%%%%%%%%%%%%%%%%%%%%
%	Some math support.					     %
%%%%%%%%%%%%%%%%%%%%%%%%%%%%%%%%%%%%%%%%%%%%%%%%%%%%%%%%%%%%%%%%%%%%%%
%
%	Theorem environments (these need the amsthm package)
%
%% \theoremstyle{plain} %% This is the default

\newtheorem{thm}{Theorem}[section]
\newtheorem{cor}[thm]{Corollary}
\newtheorem{lem}[thm]{Lemma}
\newtheorem{prop}[thm]{Proposition}
\newtheorem{ax}{Axiom}

\theoremstyle{definition}
\newtheorem{defn}{Definition}[section]

\theoremstyle{remark}
\newtheorem{rem}{Remark}[section]
\newtheorem*{notation}{Notation}

%\numberwithin{equation}{section}


%%%%%%%%%%%%%%%%%%%%%%%%%%%%%%%%%%%%%%%%%%%%%%%%%%%%%%%%%%%%%%%%%%%%%%
%	Macros.							     %
%%%%%%%%%%%%%%%%%%%%%%%%%%%%%%%%%%%%%%%%%%%%%%%%%%%%%%%%%%%%%%%%%%%%%%
%
%	Here some macros that are needed in this document:

\newcommand{\latexe}{{\LaTeX\kern.125em2%
                      \lower.5ex\hbox{$\varepsilon$}}}

\newcommand{\amslatex}{\AmS-\LaTeX{}}

\chardef\bslash=`\\	% \bslash makes a backslash (in tt fonts)
			%	p. 424, TeXbook

\newcommand{\cn}[1]{\texttt{\bslash #1}}

\makeatletter		% Starts section where @ is considered a letter
			% and thus may be used in commands.
\def\square{\RIfM@\bgroup\else$\bgroup\aftergroup$\fi
  \vcenter{\hrule\hbox{\vrule\@height.6em\kern.6em\vrule}%
                                              \hrule}\egroup}

\makeatother		% Ends sections where @ is considered a letter.
			% Now @ cannot be used in commands.

\makeindex    % Make the index

%%%%%%%%%%%%%%%%%%%%%%%%%%%%%%%%%%%%%%%%%%%%%%%%%%%%%%%%%%%%%%%%%%%%%%
% BEGIN DOCUMENT
%%%%%%%%%%%%%%%%%%%%%%%%%%%%%%%%%%%%%%%%%%%%%%%%%%%%%%%%%%%%%%%%%%%%%%
\begin{document}

\copyrightpage      % Produces the copyright page
\commcertpage       % Produces the doctoral Committee Certification of Approved Version page
\titlepage          % Produces the title page


%%%%%%%%%%%%%%%%%%%%%%%%%%%%%%%%%%%%%%%%%%%%%%%%%%%%%%%%%%%%%%%%%%%%%%
% Dedication and Acknowledgements
%%%%%%%%%%%%%%%%%%%%%%%%%%%%%%%%%%%%%%%%%%%%%%%%%%%%%%%%%%%%%%%%%%%%%%
\begin{dedication}
To my family, friends, and teachers.
\end{dedication}


\begin{acknowledgments}
I wish to thank the multitudes of people who helped me. Time would
fail me to tell of \ldots
\end{acknowledgments}


%%%%%%%%%%%%%%%%%%%%%%%%%%%%%%%%%%%%%%%%%%%%%%%%%%%%%%%%%%%%%%%%%%%%%%
% Abstract
%%%%%%%%%%%%%%%%%%%%%%%%%%%%%%%%%%%%%%%%%%%%%%%%%%%%%%%%%%%%%%%%%%%%%%
\utabstract
\indent
This document has the form of a ``fake'' doctoral dissertation in order to provide an example of such, but it is actually a
copy of Miguel Lerma's documentation for the Mathematics Department Computer Seminar of 25 March 1998 updated in July 2001
and following by Craig McCluskey to meet the March 2001 requirements of the Graduate School.


%%%%%%%%%%%%%%%%%%%%%%%%%%%%%%%%%%%%%%%%%%%%%%%%%%%%%%%%%%%%%%%%%%%%%%
% Table of Contents + List of Tables and Figures
%%%%%%%%%%%%%%%%%%%%%%%%%%%%%%%%%%%%%%%%%%%%%%%%%%%%%%%%%%%%%%%%%%%%%%
\tableofcontents
\listoftables
\listoffigures


%%%%%%%%%%%%%%%%%%%%%%%%%%%%%%%%%%%%%%%%%%%%%%%%%%%%%%%%%%%%%%%%%%%%%%
% Begin Document Body
%%%%%%%%%%%%%%%%%%%%%%%%%%%%%%%%%%%%%%%%%%%%%%%%%%%%%%%%%%%%%%%%%%%%%%

% Chapter 1 - Introduction
\include{chapter-1-introduction}

\include{chapter-introduction}

\include{chapter-instructions}

\include{chapter-howtouse}

\include{chapter-makingbib}

\include{chapter-tables+figs}

\include{chapter-math}


%%%%%%%%%%%%%%%%%%%%%%%%%%%%%%%%%%%%%%%%%%%%%%%%%%%%%%%%%%%%%%%%%%%%%%
% Appendix/Appendices
%%%%%%%%%%%%%%%%%%%%%%%%%%%%%%%%%%%%%%%%%%%%%%%%%%%%%%%%%%%%%%%%%%%%%%
%
% If you have only one appendix, use the command \appendix instead
% of \appendices.
%
\appendices

% Appendix A - Derivation of the Stern-Volmer Relationship
%%%%%%%%%%%%%%%%%%%%%%%%%%%%%%%%%%%%%%%%%%%%%%%%%%%%%%%%%%%%%%%%%%%%%%%%%%%%%%%
% Appendix A - Derivation of the Stern-Volmer Relationship
%%%%%%%%%%%%%%%%%%%%%%%%%%%%%%%%%%%%%%%%%%%%%%%%%%%%%%%%%%%%%%%%%%%%%%%%%%%%%%%

\chapter{Derivation of the Stern-Volmer Relationship}

Upon absorption of radiation, a phosphorescent molecule is excited (Figure \ref{fig:jablonski}A) from its singlet ground state ($S_0$) to an excited singlet state ($S_1$) while maintaining the pairing of its electron spins. The excited singlet state can then non-radiatively transition to an excited triplet state ($T_1$) through a process known as intersystem crossing. This results in the unpairing of its ground and excited state electron spins.

The excited triplet state molecule can return back to its singlet ground state through either radiative or non-radiative relaxation. The radiative relaxation from the excited triplet state to the singlet ground state is known as phosphorescence and occurs with a decay rate $k_{r}$. Non-radiative relaxation occurs through intersystem crossing at a decay rate $k_{n-r}$. Collision with ground triplet state molecular oxygen (\ce{^3O2}) can quench the excited triplet state molecule back to its singlet ground state while producing excited singlet state oxygen (\ce{^1O2}). This transfer of electronic excitation energy can be modeled as a first order reaction with a decay rate proportional to the product of the quenching constant ($k_{q}$) and the concentration of dissolved oxygen ([\ce{O2}]).
%
The overall rate equation can be expressed as:
%
\begin{equation}
    \frac{d[P^*]}{dt} = -(k_{r} + k_{n-r} + k_{q}[O_{2}])[P^{*}]
\end{equation}
%
where $[P^{*}]$ denotes the number of excited triplet state molecules. Assuming that $[O_{2}] >> [P^{*}]$, then the rate equation can be integrated to:
%
\begin{equation}
    [P^*] = -[P^{*}]_{0}e^{-(k_{r} + k_{n-r} + k_{q}[O_{2}])t}
\end{equation}
%
The lifetime in the presence of the quencher ($\tau$) can then be defined by inversion of the overall decay rate:
%
\begin{equation}
    \tau = \frac{1}{k_{r} + k_{n-r} + k_{q}[O_{2}]}
\end{equation}
%
In the absence of the quencher ($[O_2] = 0$), the lifetime can be simplified to:
%
\begin{equation}
    \tau_{0} = \frac{1}{k_{r} + k_{n-r}}
\end{equation}
%
and used to normalize the lifetime in the presence of the quencher:
%
\begin{equation}
    \frac{\tau_{0}}{\tau} = \frac{k_{r} + k_{n-r} + k_{q}[O_{2}]}{k_{r} + k_{n-r}}
\end{equation}
%
This can be simplified to a linear expression known as the Stern-Volmer Relationship:
%
\begin{equation}
    \frac{\tau_{0}}{\tau} = 1 + k_{q}\tau_{0}[O_{2}]
\end{equation}
%
Using Henry's Law for dissolved gases, $pO_2 = K_H[O_2]$, the molecular oxygen concentration [\ce{O2}] can be exchanged for the partial pressure of oxygen (\ce{pO2}):
%
\begin{equation}
    \frac{\tau_{0}}{\tau} = 1 + k_{q}\tau_{0}[pO_{2}]
\end{equation}
%
Lifetime can be calibrated against a standard that quantifies either the [\ce{O2}] or \ce{pO2} in a solution that matches the pH, atmospheric pressure, temperature, and salinity of the desired sample environment.


% Appendix B - Sterile Cranial Window Surgical Preparation
%%%%%%%%%%%%%%%%%%%%%%%%%%%%%%%%%%%%%%%%%%%%%%%%%%%%%%%%%%%%%%%%%%%%%%%%%%%%%%%
% Appendix B - Sterile Cranial Window Surgical Preparation
%%%%%%%%%%%%%%%%%%%%%%%%%%%%%%%%%%%%%%%%%%%%%%%%%%%%%%%%%%%%%%%%%%%%%%%%%%%%%%%

\chapter{Sterile Cranial Window Surgical Preparation}

While the optical imaging techniques described in this dissertation are non-contact, a chronic cranial window implantation is required to obtain optical access to brain tissue. This appendix details the craniotomy procedure and maintenance protocol utilized for chronic \textit{in vivo} awake imaging in mice.


%%%%%%%%%%%%%%%%%%%%%%%%%%%%%%%%%%%%%%%%%%%%%%%%%%%%%%%%%%%%%%%%%%%%%%%%%%%%%%%
% Section B.1 - Implantation of Cranial Window
%%%%%%%%%%%%%%%%%%%%%%%%%%%%%%%%%%%%%%%%%%%%%%%%%%%%%%%%%%%%%%%%%%%%%%%%%%%%%%%
\section{Implantation of Cranial Window}

Mice (CD-1, male, 25-30 g, Charles River) were anesthetized with medical air vaporized isoflurane (2.0\%) via nose-cone inhalation. Body temperature was maintained at 37 $^\circ$C with a feedback heating pad (DC Temperature Controller, FHC, Bowdoin, ME, USA). Arterial oxygen saturation, heart rate, and breath rate were monitored via pulse oximetry (MouseOx, Starr Life Sciences, Oakmont, PA, USA). After induction, mice were placed supine in a head-fixed stereotaxic frame (Narishige Scientific Instrument Lab, Tokyo, Japan) and administered carprofen (5 mg/kg, subcutaneous) for anti-inflammation and dexamethasone (2 mg/kg, intramuscular) to reduce the severity of cerebral edema following removal of the skull. Surgical instruments and artificial cerebrospinal fluid (ACSF, buffered pH 7.4) used during the craniotomy procedure were sterilized in an autoclave.

The scalp was shaved and resected to expose skull between the bregma and lambda cranial coordinates. A thin layer of cyanoacrylate (Vetbond Tissue Adhesive, 3M, St. Paul, MN, USA) was applied to the exposed skull to facilitate the adhesion of dental cement during a later step. A 2-3 mm diameter portion of the skull over the frontoparietal cortex was removed while leaving the dura intact using a dental drill (0.8 mm burr, Ideal Microdrill, Fine Science Tools, North Vancouver, BC, Canada) with regular ACSF perfusion to prevent overheating. A 5 mm round cover glass (\#1.5, World Precision Instruments, Sarasota, FL, USA) was placed over the exposed brain and a dental cement mixture was deposited along the perimeter while applying gentle pressure to the cover glass. This process bonded the cover glass to the surrounding skull to create a sterile, air-tight seal around the craniotomy and allowed for restoration of intracranial pressure. A second layer of cyanoacrylate was applied over the dental cement to further seal the cranial window. The medial and anterior edges of the window were approximately 2 mm rostral to bregma and 0.5 mm lateral to midline. Animals were allowed to recover from anesthesia and monitored for cranial window integrity and normal behavior for at least two weeks prior to imaging. Additional carprofen injections (5 mg/kg) were administered subcutaneously two, four, and seven days post-surgery to relieve inflammation from the procedure.


%%%%%%%%%%%%%%%%%%%%%%%%%%%%%%%%%%%%%%%%%%%%%%%%%%%%%%%%%%%%%%%%%%%%%%%%%%%%%%%
% Section B.2 - Headbar Attachment
%%%%%%%%%%%%%%%%%%%%%%%%%%%%%%%%%%%%%%%%%%%%%%%%%%%%%%%%%%%%%%%%%%%%%%%%%%%%%%%
\section{Headbar Attachment}

Animals designated for awake imaging underwent an additional procedure during the application of dental cement to permanently attach a metal headbar used for head fixation. The circular cutout in the headbar was aligned with the cranial window and rotated laterally until parallel with the cover glass. This ensured that the cranial window would be perpendicular to the imaging system's optical axis when the animal was restrained in the awake imaging setup. Dental cement was applied around the headbar to permanently attach it to the animal's skull.


%%%%%%%%%%%%%%%%%%%%%%%%%%%%%%%%%%%%%%%%%%%%%%%%%%%%%%%%%%%%%%%%%%%%%%%%%%%%%%%
% Section B.3 - Chronic Animal Maintenance
%%%%%%%%%%%%%%%%%%%%%%%%%%%%%%%%%%%%%%%%%%%%%%%%%%%%%%%%%%%%%%%%%%%%%%%%%%%%%%%
\section{Chronic Animal Maintenance}

Animals were checked daily to monitor both behavior and the integrity of the cranial window by veterinary staff at the University of Texas at Austin Animal Research Center (ARC). Animals were housed in climate-controlled rooms with timed lighting (12-hour light/dark cycles) to maintain a comfortable living environment and given food and water \textit{ad libitum}. Social housing with multiple animals reduced the risk of overeating commonly seen when solo housing animals. This minimized possible growth in the animal's size and helped maintain the integrity of the cranial window. Any aggression resulted in the removal of the aggressor into a separate cage.

After several weeks of recovery, animals were used for both acute and chronic imaging experiments. Cranial windows were lightly cleaned prior to each imaging session using a cotton swab and 70\% ethanol (v/v). If necessary, a topical application of mineral oil was used to improve image quality by index matching. Any discoloration on or around the cranial window was documented and monitored for possible infection. Any cracks in or breaking of the cranial window were also documented and resulted in the immediate euthanasia of the animal.


%%%%%%%%%%%%%%%%%%%%%%%%%%%%%%%%%%%%%%%%%%%%%%%%%%%%%%%%%%%%%%%%%%%%%%%%%%%%%%%
% END Appendix B
%%%%%%%%%%%%%%%%%%%%%%%%%%%%%%%%%%%%%%%%%%%%%%%%%%%%%%%%%%%%%%%%%%%%%%%%%%%%%%%



\include{chapter-appendix1}
\include{chapter-appendix2}
\include{chapter-appendix3}


%%%%%%%%%%%%%%%%%%%%%%%%%%%%%%%%%%%%%%%%%%%%%%%%%%%%%%%%%%%%%%%%%%%%%%
% Bibliography
%%%%%%%%%%%%%%%%%%%%%%%%%%%%%%%%%%%%%%%%%%%%%%%%%%%%%%%%%%%%%%%%%%%%%%
\bibliographystyle{unsrt}
\bibliography{bibliography.bib}


%%%%%%%%%%%%%%%%%%%%%%%%%%%%%%%%%%%%%%%%%%%%%%%%%%%%%%%%%%%%%%%%%%%%%%
% Vita
%%%%%%%%%%%%%%%%%%%%%%%%%%%%%%%%%%%%%%%%%%%%%%%%%%%%%%%%%%%%%%%%%%%%%%
\begin{vita}
Colin Tan Sullender was born in Birmingham, Alabama in 1988 to Wayne Sullender and Thuan Hong Tan. He graduated from Homewood High School in Homewood, Alabama in 2007. He pursued a Bachelor of Science in bioengineering at the University of Washington in Seattle and graduated in 2011 with College Honors in Bioengineering. As an undergraduate, he worked as a research assistant in the laboratory of Dr. Wendy E. Thomas. He began his doctorate in biomedical engineering at The University of Texas at Austin in 2011 in the laboratory of Dr. Andrew K. Dunn working on the development of novel optical imaging systems for studying ischemic stroke. He completed his Master of Science in Engineering in biomedical engineering in 2016 and completed his doctoral degree in August 2018.
\end{vita}


%%%%%%%%%%%%%%%%%%%%%%%%%%%%%%%%%%%%%%%%%%%%%%%%%%%%%%%%%%%%%%%%%%%%%%
% END DOCUMENT
%%%%%%%%%%%%%%%%%%%%%%%%%%%%%%%%%%%%%%%%%%%%%%%%%%%%%%%%%%%%%%%%%%%%%%
\end{document}
