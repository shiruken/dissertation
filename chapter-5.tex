%%%%%%%%%%%%%%%%%%%%%%%%%%%%%%%%%%%%%%%%%%%%%%%%%%%%%%%%%%%%%%%%%%%%%%%%%%%%%%%
% Chapter 5 - Chronic Awake Imaging of Photothrombotic Stroke
%%%%%%%%%%%%%%%%%%%%%%%%%%%%%%%%%%%%%%%%%%%%%%%%%%%%%%%%%%%%%%%%%%%%%%%%%%%%%%%

\chapter{Chronic Awake Imaging of Photothrombotic Stroke} \label{ch:awake}

Anesthesia is widely used in preclinical neuroscience research to sedate animals while imaging despite systemic effects on neuronal and vascular function \cite{Janssen:2004ih}. The inhalation anesthetic isoflurane has been shown to reduce excitatory synaptic transmission \cite{BergJohnsen:1992wk}, impair oxygen autoregulation \cite{Aksenov:2012wh}, suppress the magnitude and speed of neurovascular coupling \cite{Pisauro:2013cx}, and cause abnormal increases in CBF \cite{Strebel:1995uh}. Isoflurane has also been shown to convey neuroprotective effects that may reduce the severity of ischemic lesions \cite{Sakai:2007wc, Burchell:2013tj} and suppress the occurrence and frequency of spreading depolarizations \cite{Kudo:2016ho}. These effects can mask the benefits of neuroprotective agents and potentially impact the outcomes of long-term studies \cite{Kapinya:ua, Seto:2014ga}. Our lab has reported \cite{Ponticorvo:2010uv, Kazmi:2013ey, Sullender:2018ff} conflicting vascular \ce{pO2} measurements that can likely be attributed to the use of different general anesthetics (urethane vs. isoflurane).

The elimination of anesthesia from neuroimaging experiments has grown increasingly popular in recent years with two primary strategies taking the forefront. The first is the usage of head-mounted microscopes that miniaturize many of the optical components into a package that can be installed on the head of a mouse \cite{Gu:1999ky, Helmchen:2001tw, Flusberg:2005tq}. While this technique allows for freely-moving tethered subjects, it requires extensive optical engineering and introduces significant motion artifacts caused by normal animal behavior \cite{Helmchen:2001tw}. Miniaturizing the system described in this dissertation would require significant sacrifices in spatial and temporal resolutions for both imaging modalities.

The second strategy involves restraining the animal's head while positioned on a rotating treadmill \cite{Pisauro:2013cx, Dombeck:2007gr, Wienisch:2011ju} or confined in a small chamber \cite{Silasi:2016dq}, allowing the use of existing imaging platforms. This technique allows for walking or running in place while minimizing motion of the head and imaging region. A spherical treadmill design has been used to perform awake two-photon microscopy with only an estimated 2-5 $\mu$m of lateral motion \cite{Dombeck:2007gr}, which is near the resolution of the dual-modality imaging system. This chapter details the transition from anesthetized to awake animal imaging, including the selection of a treadmill-based restraint system and \textit{in vivo} demonstrations of chronic awake imaging.



%%%%%%%%%%%%%%%%%%%%%%%%%%%%%%%%%%%%%%%%%%%%%%%%%%%%%%%%%%%%%%%%%%%%%%%%%%%%%%%
% Section 5.1 - Awake Imaging System Designs
%%%%%%%%%%%%%%%%%%%%%%%%%%%%%%%%%%%%%%%%%%%%%%%%%%%%%%%%%%%%%%%%%%%%%%%%%%%%%%%
\section{Awake Imaging System Designs}

\blindtext



%%%%%%%%%%%%%%%%%%%%%%%%%%%%%%%%%%%%%%%%%%%%%%%%%%%%%%%%%%%%%%%%%%%%%%%%%%%%%%%
% Section 5.2 - Awake vs. Anesthetized Measurements
%%%%%%%%%%%%%%%%%%%%%%%%%%%%%%%%%%%%%%%%%%%%%%%%%%%%%%%%%%%%%%%%%%%%%%%%%%%%%%%
\section{Awake vs. Anesthetized Measurements}

\blindtext



%%%%%%%%%%%%%%%%%%%%%%%%%%%%%%%%%%%%%%%%%%%%%%%%%%%%%%%%%%%%%%%%%%%%%%%%%%%%%%%
% Section 5.3 - Awake Targeted Photothrombosis Induction
%%%%%%%%%%%%%%%%%%%%%%%%%%%%%%%%%%%%%%%%%%%%%%%%%%%%%%%%%%%%%%%%%%%%%%%%%%%%%%%
\section{Awake Targeted Photothrombosis Induction}

\blindtext



%%%%%%%%%%%%%%%%%%%%%%%%%%%%%%%%%%%%%%%%%%%%%%%%%%%%%%%%%%%%%%%%%%%%%%%%%%%%%%%
% Section 5.4 - Chronic Awake Post-Stroke Hemodynamics
%%%%%%%%%%%%%%%%%%%%%%%%%%%%%%%%%%%%%%%%%%%%%%%%%%%%%%%%%%%%%%%%%%%%%%%%%%%%%%%
\section{Chronic Awake Post-Stroke Hemodynamics}

\blindtext



%%%%%%%%%%%%%%%%%%%%%%%%%%%%%%%%%%%%%%%%%%%%%%%%%%%%%%%%%%%%%%%%%%%%%%%%%%%%%%%
% Section 5.5 - Discussion
%%%%%%%%%%%%%%%%%%%%%%%%%%%%%%%%%%%%%%%%%%%%%%%%%%%%%%%%%%%%%%%%%%%%%%%%%%%%%%%
\section{Discussion}

\blindtext



%%%%%%%%%%%%%%%%%%%%%%%%%%%%%%%%%%%%%%%%%%%%%%%%%%%%%%%%%%%%%%%%%%%%%%%%%%%%%%%
% END Chapter 5
%%%%%%%%%%%%%%%%%%%%%%%%%%%%%%%%%%%%%%%%%%%%%%%%%%%%%%%%%%%%%%%%%%%%%%%%%%%%%%%
