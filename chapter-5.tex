%%%%%%%%%%%%%%%%%%%%%%%%%%%%%%%%%%%%%%%%%%%%%%%%%%%%%%%%%%%%%%%%%%%%%%%%%%%%%%%
% Chapter 5 - Chronic Awake Imaging of Photothrombotic Stroke
%%%%%%%%%%%%%%%%%%%%%%%%%%%%%%%%%%%%%%%%%%%%%%%%%%%%%%%%%%%%%%%%%%%%%%%%%%%%%%%

\chapter{Chronic Awake Imaging of Photothrombotic Stroke} \label{ch:awake}

Anesthesia is widely used in preclinical neuroscience research to sedate animals while imaging despite systemic effects on neuronal and vascular function \cite{Janssen:2004ih}. The inhalation anesthetic isoflurane has been shown to reduce excitatory synaptic transmission \cite{BergJohnsen:1992wk}, impair oxygen autoregulation \cite{Aksenov:2012wh}, suppress the magnitude and speed of neurovascular coupling \cite{Pisauro:2013cx}, and cause abnormal increases in CBF \cite{Strebel:1995uh}. Isoflurane has also been shown to convey neuroprotective effects that may reduce the severity of ischemic lesions \cite{Sakai:2007wc, Burchell:2013tj} and suppress the occurrence and frequency of spreading depolarizations \cite{Kudo:2016ho}. These effects can mask the benefits of neuroprotective agents and potentially impact the outcomes of long-term studies \cite{Kapinya:ua, Seto:2014ga}. Our lab has reported \cite{Ponticorvo:2010uv, Kazmi:2013ey, Sullender:2018ff} conflicting vascular \ce{pO2} measurements that can likely be attributed to the use of different general anesthetics (urethane vs. isoflurane).

The elimination of anesthesia from neuroimaging experiments has grown increasingly popular in recent years with two primary strategies taking the forefront. The first is the usage of head-mounted microscopes that miniaturize many of the optical components into a package that can be installed on the head of a mouse \cite{Gu:1999ky, Helmchen:2001tw, Flusberg:2005tq}. While this technique allows for freely-moving tethered subjects, it requires extensive optical engineering and introduces significant motion artifacts caused by normal animal behavior \cite{Helmchen:2001tw}. Miniaturizing the system described in this dissertation would require significant sacrifices in spatial and temporal resolutions for both imaging modalities.

The second strategy involves restraining the animal's head while positioned on a rotating treadmill \cite{Pisauro:2013cx, Dombeck:2007gr, Wienisch:2011ju, Kaifosh:2013fy, Heiney:2018gq} or confined in a small chamber \cite{Silasi:2016dq}, permitting the use of existing imaging platforms. This technique allows for walking or running in place while minimizing motion of the head and imaging region. A spherical treadmill design has been used to perform awake two-photon microscopy with only an estimated 2-5 $\mu$m of lateral motion \cite{Dombeck:2007gr}, which is near the resolution of the dual-modality imaging system. This chapter details the transition from anesthetized to awake animal imaging, including the selection of a treadmill-based restraint system and \textit{in vivo} demonstrations of chronic awake imaging.



%%%%%%%%%%%%%%%%%%%%%%%%%%%%%%%%%%%%%%%%%%%%%%%%%%%%%%%%%%%%%%%%%%%%%%%%%%%%%%%
% Section 5.1 - Awake Imaging System Design
%%%%%%%%%%%%%%%%%%%%%%%%%%%%%%%%%%%%%%%%%%%%%%%%%%%%%%%%%%%%%%%%%%%%%%%%%%%%%%%
\section{Awake Imaging System Design}

The spherical treadmill described by Dombeck \textit{et al.} \cite{Dombeck:2007gr} is a popular design in the neuroscience community. It features a large (8-inch) Styrofoam ball supported on an air cushion produced by a perforated hemispheric casting. The ball allows for two dimensions of free movement, which reduces the torque the animal can apply to its permanently-attached metal headbar that encircles the cranial window. An implementation of this design using a 3D-printed base can be seen in Figure \ref{fig:awakesystems}A. A series of channels spanning the plastic cylinder direct the flow of air upwards to form the supporting air cushion. Flow is regulated using the standard house air supply and can be adjusted as needed depending on the weight of the ball and the subject. While preliminary tests showed the design was functional, the loudness of the flowing air was deemed unsuitable for practical use. The height of the entire apparatus was also a major limitation, with only the dual-modality imaging system capable of the necessary vertical translation.

% Figure - Awake System Designs
\begin{figure}
    \includegraphics{figures/chapter_5/awakesystems.pdf}
    \caption{
        \label{fig:awakesystems}
        \textbf{(A)} Air-cushioned spherical treadmill, \textbf{(B)} foam wheel treadmill, and \textbf{(C)} low-profile continuous belt treadmill. Close-up of the \textbf{(D)} front and \textbf{(E)} rear pulley wheels on the belt treadmill. \textbf{(F)} Mouse head-restrained on the belt treadmill using the permanently attached headbar.
    }
\end{figure}

Two alternative designs with lower profiles were examined. The first implemented (Figure \ref{fig:awakesystems}B) a rotating foam wheel \cite{Heiney:2018gq} that allowed for one dimension of free movement. The coarse surface of the foam appeared to offer a better grip for the animal than the Styrofoam ball, making it easier to walk. While the vertical height was still substantially greater than the stereotaxic frame used during anesthetized imaging, the apparatus could easily fit under both the dual-modality and standalone MESI systems. However, issues were encountered with the stability of the rotation caused by difficulties in precisely centering the axle through the foam. These rotational variations in the position of the surface of the wheel relative to the headbar caused large motion artifacts in the resulting speckle contrast imagery despite head fixation.

The second design implemented (Figure \ref{fig:awakesystems}C) a continuous belt treadmill \cite{Royer:2012gw, Kaifosh:2013fy} that also allowed for one dimension of self-propelled movement. The rubberized tread rotated over two pulleys made from LEGO tires and axles (Figure \ref{fig:awakesystems}D-E) with the animal centered over a flat piece of plastic. This eliminated the vertical variations that negatively impacted the wheel-based design and resulted in more stable LSCI measurements when the animal moved. The height of the treadmill was also further reduced and only slightly taller than the conventional stereotaxic frame. Figure \ref{fig:awakesystems}F depicts a mouse restrained on the treadmill system via its permanently attached metal headbar (see Appendix \ref{app:headbar_attachment}).



%%%%%%%%%%%%%%%%%%%%%%%%%%%%%%%%%%%%%%%%%%%%%%%%%%%%%%%%%%%%%%%%%%%%%%%%%%%%%%%
% Section 5.2 - Effects of Anesthesia on Hemodynamics
%%%%%%%%%%%%%%%%%%%%%%%%%%%%%%%%%%%%%%%%%%%%%%%%%%%%%%%%%%%%%%%%%%%%%%%%%%%%%%%
\section{Effects of Anesthesia on Hemodynamics}

\textit{In vivo} imaging with the head-restrained treadmill system was demonstrated by comparing awake and anesthetized hemodynamics. The subject was mounted on the treadmill using its headbar and allowed to acclimate for 10-15 minutes prior to imaging. The mounting rods were adjusted as necessary to maintain a comfortable head position and to orient the cranial window perpendicularly to the optical axis. Subjects quickly adapted to walking on the treadmill and would exhibit regularly grooming behavior when stationary.

%%%%%%%%%%%%%%%%%%%%%%%%%%%%%%%%%%%%%%%%%%%%%%%%%%%%%%%%%%%%%%%%%%%%%%%%%%%%%%%
\subsection{Dynamic Cerebral Blood Flow} \label{ssec:dynamicawakecbf}

An extended MESI acquisition was performed on the standalone MESI system to monitor the stability of awake and anesthetized measurements and the transition between the two states. The awake subject was imaged continuously for one hour with a nose-cone placed directly in front of the restrained animal and delivering only medical air. General anesthesia was then quickly induced with medical air vaporized isoflurane (2.5\%) via the nose-cone. After several minutes, the isoflurane was reduced to 1.5\% and a feedback heating pad (DC Temperature Controller, FHC) placed under the animal to regulate body temperature. The anesthetized subject was imaged for an additional one hour before being removed from anesthesia and allowed to recover on a heating pad.

% Figure - Speckle Awake vs. Anesthetized
\begin{figure}
    \includegraphics{figures/chapter_5/speckleawakeanes.pdf}
    \caption{
        \label{fig:speckleawakeanes}
        Speckle contrast (top) and MESI ICT (bottom) imagery from a mouse while awake (left) and anesthetized (right) exhibits the systemic change in CBF. (Scale bar = 1 mm).
    }
\end{figure}

Figure \ref{fig:speckleawakeanes} depicts averaged single-exposure (5 ms) speckle contrast images and their corresponding MESI ICT images during the awake and anesthetized states. Anesthesia caused a systematic increase in CBF as exhibited by the decrease in average speckle contrast (0.199 $\to$ 0.166) and increase in average ICT (2798 $s^{-1} \to$ 6579 $s^{-1}$). Significant vasodilation can also be seen in the large central vein and many smaller vessels only become visible in the anesthetized state. These results are consistent with volatile anesthetics being potent vasodilators that cause dose-dependent increases in baseline CBF uncoupled from local metabolic demands \cite{Masamoto:2012bj}.

Figure \ref{fig:flowawakeanes}A highlights two vascular regions (R1, R2) and one parenchyma region (R3) analyzed over the entire awake-to-anesthetized MESI acquisition. While the head-restraint minimized most undesired motion, \texttt{elastix} was used during post-processing to rigidly align all speckle contrast frames to the beginning of the acquisition. Figure \ref{fig:flowawakeanes}B depicts the resulting relative blood flow timecourses using the average of the entire one hour of awake imaging as the ICT baseline. Flow was stable throughout the awake imaging segment with occasional spikes corresponding to animal motion. However, it is difficult to determine whether these are real changes in blood flow or if the physical motion itself is causing the change in $\tau_c$. The induction of general anesthesia rapidly caused a 3x increase in relative flow across all three ROIs before reaching a semi-stable state after about 10 minutes. The spike around $t$ = 5000 seconds was likely caused by an unstable anesthesia plane or difficulty with breathing resulting in increased animal motion. However, this quickly subsides and the flow remains relatively stable as it slightly declines to around 2.5x of baseline in all three ROIs over the remainder of the imaging session.

% Figure - Awake to Anesthetized Timecourses
\begin{figure}
    \includegraphics{figures/chapter_5/flowawakeanes.pdf}
    \caption{
        \label{fig:flowawakeanes}
        \textbf{(A)} MESI ICT was calculated within two branches of a large central vein (R1 and R2) and a parenchyma region (R3) as a mouse was anesthetized (Scale bar = 1 mm). \textbf{(B)} Relative ICT timecourses over one hour of awake imaging followed by one hour of anesthetized imaging. \textbf{(C)} Estimated diameters of R1 and R2 over the imaging session. \textbf{(D)} Relative ICT timecourses using the multiple dynamic scattering MESI model.
    }
\end{figure}

In order to examine anesthesia-induced vasodilation, the diameters of both vessels (R1, R2) were estimated from the speckle contrast images. Cross-sectional profiles (the black bars in Figure \ref{fig:flowawakeanes}A) averaged over the length of each ROI were fitted to a Gaussian distribution. The vessel diameter was defined as the full width at half maximum (FWHM) of the fitted cross-section. The FWHM was computed for both vessels in every frame of the acquired data and is shown in Figure \ref{fig:flowawakeanes}C. Similar to the relative flow within the ROIs, the diameters remained stable throughout the entire awake imaging segment. However, the induction of anesthesia caused significant vasodilation in vessel R1 (+54\%) while vessel R2 increased slightly (+23\%) over the entire anesthetized imaging segment. The reason for this discrepancy is unclear, especially since the imagery in Figure \ref{fig:speckleawakeanes} appears to show systemic vasodilation. The use of speckle contrast images for estimating the vessel diameter is a somewhat ill-posed problem since the technique has reduced spatial resolution compared to a conventional reflectance image.

As discussed in Chapter \ref{ch:mesi}, the standard MESI model represented by Equation \ref{eq:mesi} assumes that detected photons only experience single dynamic scattering interactions and the underlying particle motion has a Lorentzian velocity distribution indicative of Brownian motion \cite{Parthasarathy:2008el}. However, camera pixels corresponding to resolvable surface vasculature are more likely to sample photons that have experienced multiple dynamic scattering events \cite{Davis:2014kc} and exhibit Gaussian velocity distributions indicative of bulk flow \cite{Kazmi:2015du}. The single dynamic scattering correlation time ($\tau_c^{sds}$) can be converted to the multiple dynamic scattering correlation time ($\tau_c^{mds}$) by scaling the former by the average number of dynamic scattering events ($N_d$), which is proportional to the vessel diameter \cite{Kazmi:2015du}. Figure \ref{fig:flowawakeanes}D plots the ``corrected'' relative ICT$_{mds}$ that accounts for multiple scattering events using the vessel width data from Figure \ref{fig:flowawakeanes}C. This new estimate shows a slightly reduced 2.5x increase in flow during anesthesia in the vascular ROIs compared to the 3x increase seen with rICT$_{sds}$. It also reveals a more substantial decrease in flow over the remainder of the anesthetized imaging session down to only 1.5x of baseline.

%%%%%%%%%%%%%%%%%%%%%%%%%%%%%%%%%%%%%%%%%%%%%%%%%%%%%%%%%%%%%%%%%%%%%%%%%%%%%%%
\subsection{Static Cerebral Blood Flow and Oxygen Tension}

MESI and \ce{pO2} measurements were performed on the dual-modality imaging system to examine the behavior of the two hemodynamic parameters in the awake and anesthetized states. The subject was briefly anesthetized to administer Oxyphor PtG4 via retro-orbital injection for a target blood plasma concentration of 5 $\mu$M and allowed to recover for two hours before being placed on the treadmill. Awake MESI and 445 nm \ce{pO2} measurements were acquired from the ROIs shown in Figure \ref{fig:specklepO2awakeanes}A covering one arteriole (A1), three veins (V1, V2, V3), and one parenchyma region (P1). The subject was then anesthetized in place via nose cone inhalation of medical air vaporized isoflurane (1.5\%) and another set MESI and 445 nm \ce{pO2} measurements acquired after 15 minutes.

% Figure - Awake and Anesthetized Static MESI ICT and pO2
\begin{figure}
    \includegraphics{figures/chapter_5/specklepO2awakeanes.pdf}
    \caption{
        \label{fig:specklepO2awakeanes}
        \textbf{(A)} Awake single-exposure speckle contrast image overlaid with regions targeted for ICT and \ce{pO2} measurements (Scale bar = 1 mm). One arteriole (A1), three veins (V1, V2, V3), and one parenchyma region (P1) were examined. The ROIs covered areas between 0.028 - 0.039 mm$^2$. \textbf{(B)} MESI ICT and \textbf{(C)} 445 nm \ce{pO2} measurements within the targeted regions during awake and anesthetized states (mean $\pm$ s.d.).
    }
\end{figure}

Figure \ref{fig:specklepO2awakeanes}B shows the change in MESI ICT within each of the ROIs between the awake and anesthetized states. Similar to the results shown in Section \ref{ssec:dynamicawakecbf}, there was a 2.5x increase in relative flow while under general anesthesia. The \ce{pO2} measurements also experienced a systematic increase when in the anesthetized state (Figure \ref{fig:specklepO2awakeanes}C). The awake \ce{pO2} is significantly lower than the values reported in Section \ref{sec:acutedemo} and the arteriole only exhibits a marginally higher \ce{pO2} compared to the other regions. These results are similar to the findings of Lyons \textit{et al.} \cite{Lyons:2016bd}, who reported a mean capillary \ce{pO2} of 36 mmHg in the cortex and suggested the increase in \ce{pO2} during anesthesia is caused by the increase in blood flow.



%%%%%%%%%%%%%%%%%%%%%%%%%%%%%%%%%%%%%%%%%%%%%%%%%%%%%%%%%%%%%%%%%%%%%%%%%%%%%%%
% Section 5.3 - Awake Targeted Photothrombosis Induction
%%%%%%%%%%%%%%%%%%%%%%%%%%%%%%%%%%%%%%%%%%%%%%%%%%%%%%%%%%%%%%%%%%%%%%%%%%%%%%%
\section{Awake Targeted Photothrombosis Induction}

\blindtext



%%%%%%%%%%%%%%%%%%%%%%%%%%%%%%%%%%%%%%%%%%%%%%%%%%%%%%%%%%%%%%%%%%%%%%%%%%%%%%%
% Section 5.4 - Chronic Awake Post-Stroke Hemodynamics
%%%%%%%%%%%%%%%%%%%%%%%%%%%%%%%%%%%%%%%%%%%%%%%%%%%%%%%%%%%%%%%%%%%%%%%%%%%%%%%
\section{Chronic Awake Post-Stroke Hemodynamics}

\blindtext



%%%%%%%%%%%%%%%%%%%%%%%%%%%%%%%%%%%%%%%%%%%%%%%%%%%%%%%%%%%%%%%%%%%%%%%%%%%%%%%
% Section 5.5 - Discussion
%%%%%%%%%%%%%%%%%%%%%%%%%%%%%%%%%%%%%%%%%%%%%%%%%%%%%%%%%%%%%%%%%%%%%%%%%%%%%%%
\section{Discussion}

\blindtext



%%%%%%%%%%%%%%%%%%%%%%%%%%%%%%%%%%%%%%%%%%%%%%%%%%%%%%%%%%%%%%%%%%%%%%%%%%%%%%%
% END Chapter 5
%%%%%%%%%%%%%%%%%%%%%%%%%%%%%%%%%%%%%%%%%%%%%%%%%%%%%%%%%%%%%%%%%%%%%%%%%%%%%%%
